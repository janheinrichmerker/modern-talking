\documentclass[english,handout]{mlutalk}

\title{Ideas for Approaches -- Modern Talking}
% \title{%
%   Modern Talking: Key-Point Analysis \\
%   using Modern Natural Language Processing
% }
\subtitle{Natural Language Processing, Summer Semester 2021}
\author{Max Henze \and Hanh Luu \and Jan Heinrich Reimer}
\institute{Martin Luther University Halle-Wittenberg}
\date{\today}
\titlegraphic{\includegraphics[width=3cm]{figures/mlu-halle}}

\addbibresource{../literature/literature.bib}

\usepackage{listings}
\usepackage{xspace}
\usepackage{tabularx}
\usepackage{booktabs}

\newcommand{\Bert}{\textsc{Bert}\xspace}
\newcommand{\ArgKP}{\mbox{ArgKP}\xspace}
\newcommand{\ArgQ}{\mbox{IBM-ArgQ-Rank-30kArgs}\xspace}
\newcommand{\BiLSTM}{\mbox{BiLSTM}\xspace}
\newcommand{\BertBase}{\textsc{Bert}-Base\xspace}
\newcommand{\BertLarge}{\textsc{Bert}-Large\xspace}
\newcommand{\DistilBert}{Distil\textsc{Bert}\xspace}
\newcommand{\TF}{\mbox{TF}\xspace}
\newcommand{\TFIDF}{\mbox{TF/IDF}\xspace}

\lstset{basicstyle=\ttfamily\footnotesize}

\begin{document}

\titleframe

\begin{frame}{Term Overlap (Rule-Based)}
  \begin{itemize}
    \item count words that occur in argument and key Point
    \item in relation to minimal length \(\to\) normalized score
    \item preprocessing: Snowball stemmer~\cite{Porter1980}
    \item only decide if relative overlap \(\leq 0.4\) or \(\geq 0.6\)
  \end{itemize}

  \begin{block}{Pro/Con}
    \begin{itemize}
      \pro simple baseline
      \pro reasonable improvement with preprocessing
      \con poor performance on shared task's strict metric
    \end{itemize}
  \end{block}
\end{frame}

\begin{frame}{Using Language Models}

  \begin{itemize}
    \item Using \Bert we can create the following input structure \\
    \lstinline{[CLS] Argument [SEP] Key Point}
    \item \Bert allows learning contextual relations between words
  \end{itemize}

  \begin{table}
  \begin{tabular}{ll}
    \toprule
    \textbf{Pro} & \textbf{Con} \\
    \midrule
    often used as baseline & computationally intensive \\
    good results (though not too good) & might need to use \DistilBert \\
    \bottomrule
  \end{tabular}
  \end{table}

\end{frame}

\begin{frame}{Topic Information / \BiLSTM}
  
  \begin{itemize}
    \item Stab et al. show that integrating topic information \textquote{has a strong impact on argument prediction}~\cite{StabMSRG2018}
    \item Therefore maybe use \BiLSTM and integrate topic information only on \(i\)~and \(c\)~gates
  \end{itemize}

\end{frame}

\appendix

\section{\appendixname}

\bibliographyframe

\end{document}
